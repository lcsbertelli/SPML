\chapter{Conclusion}
\label{sec:conclusion}
In this thesis, we have created a secure privacy-preserving machine learning framework SPML that enforces privacy, confidentiality, and integrity on any native machine/deep learning system. We have developed this framework with the help of Google TensorFlow \cite{24}, differential privacy library \cite{11}, or randomized response to achieve privacy, Intel SGX \cite{9} to achieve security goals of confidentiality and integrity and SCONE \cite{22} for integration with TEEs. We have used the SCONE platform so that native and legacy systems can use SPML with minimum code changes.

Privacy property is achieved by adding noise or randomized response and for enabling security property we have used Intel SGX. For noise, SPML shows a trend for the accuracy upon enabling privacy property, which is accuracy increases with epsilon value and this trend remains unaffected upon enabling security property for both the training and inference phase. The latency however increases upon enabling privacy and security property for the training phase. On the other hand, the inference phase for SPML can be completed within a few seconds upon enabling these properties. 

The latency for the randomized response mechanism is less as compared to using a noise mechanism. However, accuracy for randomized response fluctuates a lot which makes it less useful in practice. On the other hand, due to the notion that we saw for noise between the accuracy and epsilon values in the evaluation section ~\ref{sec:eval}, which is accuracy increases with epsilon values, using noise is much more practical as compared to the randomized response. 

It can be concluded that the SPML system is practical to use for inference but for training, it takes more time. If we have to make our native machine learning system privacy-preserving and secure then we have to bear this cost in terms of latency in both the mechanisms i.e randomized response or noise. We can use some of the advanced features to deploy the system and reduce training time further such as 'Inference only in SCONE hardware mode' ~\ref{sec:ioonly}, 'Save and load model' ~\ref{sec:workaroundSLmodel} or 'federated learning' ~\ref{sec:flTEE} as discussed in the evaluation chapter.

Our system SPML, not only makes any native existing machine/deep learning system privacy-preserving but it also provides confidentiality and integrity using Intel SGX. We have used the SCONE runtime environment to wrap up the integration with TEEs and SCONE being a container-based environment makes SPML easy to deploy, run, and maintain in native or cloud environments. SPML is not only easy to use for new developments but any existing native machine learning system can use our system SPML framework and with minimum code changes privacy and security properties can be added in these native machine learning systems.